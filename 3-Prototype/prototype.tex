\chapter{Prototypical Implementation}
\label{ch:prototype}

\vspace{-1cm}
\begin{center}
Eduard Hirsch
\end{center}

\section{Architecture}

TODO

\section{Prototype}

The prototype is a block-chain realisation of INTERLACE, can be found on GitHub\footnote{https://github.com/InterlaceProject/InterlaceBlockchain} and is based on the specifications created in deliverable D3.1 as well as the ASIM Specification of the requirements.

First it is necessary to install the pre-requisites which are available for Linux and Mac OS. Currently these are the recommended operating systems, however, with additional effort it might be possible to run the INTERLACE blockchain on Windows directly. To support windows user a virtual machine set-up is also available.

Additionally, it is also important to set-up a development environment described at the composer GitHub repository. If you do not want to set up the complete environment it would be still recommended to install and start Composer Playground. Playground enables you to connect, alter and test the INTERLACE payment network. Nevertheless, playground is not required and you might use composer-cli or other methods to utilized the network.

\subsection{Install}

This part of the documents talks about how to set-up and run the business network on your machine. However, before you actually can begin you need to install the pre-requisites which are listed at the hyperledger composer documentation\footnote{https://hyperledger.github.io/composer/latest/installing/installing-prereqs.html}. There they are also providing a script for installing all the requirements for your machine.

However, if you are a windows user those script won't help because for now most of the packages are not prepared for a windows operating system. For this user group a virtual machine has been prepared which is also installing all the necessary frameworks and software tools. This virtual machine is controlled by vagrant\footnote{https://www.vagrantup.com/} and uses hyper-v or virtual box as hypervisor. That VM configuration also is published at GitHub\footnote{https://github.com/hirsche/hyperledger}.

\textbf{Environment Start-Up}

Once the hyperledger environment is installed, the next step is about starting the INTERLACE environment. To make communication uniform the block-chain is configured to publish all services under the host name "interlace.chain". If you are windows user and used the suggested vagrant set-up the new hostname is added to your host-file during start-up time if not you'd need to configure the name by yourself.

\textbf{Configure Hostnames}

Thus before executing it is important for your local set-up to add a host name entry for "interlace.chain". Usually this entry will point to ip 127.0.0.1 (localhost). However, on a production system or if you decide to start the hyperledger composer services on different interface, fix the IP accordingly. Here is a list of hosts-file locations according to your operating systems

\begin{itemize}
	\item Mac OS: /private/etc/hosts
    \item Linux: /etc/hosts
    \item Windows: C:\textbackslash Windows\textbackslash System32\textbackslash drivers\textbackslash etc\textbackslash hosts
\end{itemize}

The format may vary a little but usually a new host with its hostname is defined using it's IP and the desired host name like

\begin{lstlisting}
	127.0.0.1        interlace.chain
\end{lstlisting}

Depending on the operating system it might be also necessary to update and restart the respective services.

\textbf{Run fabric block chain (the first time)}

Now the main configurations have been done and hyperledger fabric can be started, which acts as a base for hyperledger composer. Thus, now the GitHub repository \footnote{https://github.com/InterlaceProject/InterlaceBlockchain} if not yet done may be downloaded by using the git versioning system calling

\begin{lstlisting}[language=bash]
	git clone https://github.com/InterlaceProject/InterlaceBlockchain.git
\end{lstlisting}

In the created directory "InterlaceBlockchain" the business network implementation including a web application can be found. The next listing shows the bash script which downloads the fabric docker container and finally starts the container using docker-composer \footnote{https://docs.docker.com/compose/}:

\begin{lstlisting}[language=bash]
	cd fabric
	./downloadFabric.sh # updates images - only the first time necessary
	./startFabric.sh # start up docker environment using docker-compose
\end{lstlisting}

\textbf{Initialize Interlace-Chain}

Finally, after fabric has been started it is necessary to initialize the block chain with a call of

\begin{lstlisting}[language=bash]
	cd chain
	./initNetwork.sh # use hyperledger composer to create a business network and deploy it
\end{lstlisting}

\textbf{./initNetwork.sh} will copy all models and script to the network peers to make them accessible in the hyperledger blockchain.

You may further use playground to access and test Credit- or DebitTransfer transactions. \textbf{data.json} should act as a helper to init the network by hand, but it is recommended to update the JavaScript function \textit{initBlockchain(transfer)} in \textit{./chain/lib/init.js}. That chain-code part is executed when transaction InitBlockchain is submitted. Be careful to run \textbf{InitBlockchain} only once otherwise errors or duplicate entries might happen.

\textbf{Network updates after chain-code changes}

After changes to the acl, cto, queries, the libraries or other parts of the core chain-code application the network needs to be updated. This can be achieved by executing

\begin{lstlisting}[language=bash]
	./chain/updateNetwork.sh
\end{lstlisting}

This script reads the current version number of \textit{package.json} file increases it by one and creates a new bna package. When scripts are correct and the bna-package could be created it is deployed to the peers and the network updated to the new network version which will utilize the new bna package.

\textbf{Shutting down}

Sometimes it is useful to throw away everything and restart from scratch. To teardown fabric and remove card left overs execute:

\begin{lstlisting}[language=bash]
	cd fabric
	./teardownFabric.sh
	./deletePlaygroundCards.sh
\end{lstlisting}

\textbf{Start a rest server}

Once the network is running (no playground needed) it is also possible to start a HTTP-Server which allows to interact with the network over REST. The script

\begin{lstlisting}[language=bash]
	cd chain
	./startRestServer.sh
\end{lstlisting}

starts the server and allows to get an overview of the restful interface by opening

\begin{lstlisting}
http://interlace.chain:3000/explorer
\end{lstlisting}

in a browser. The REST interface itself may be contacted over

\begin{lstlisting}
http://interlace.chain:3000/
\end{lstlisting}

when using it together with an external application. In case you didn't set-up the host interlace.chain in your hosts file and you are running all the services locally without a VM you might also use localhost instead of interlace.chain as host name.

\subsection{Working with the environment}

Next a closer look is taken on how the environment might be facilitated using different approaches. It is possible to connect to the chain using composer-cli, taking advantage of composer playground (the graphical interface) or use the simple web-front-end created for the project.

\textbf{Start and test network with playground}

If you've decided to install and use Composer Playground it can be started using that command

\begin{lstlisting}[language=bash]
	composer-playground
\end{lstlisting}

The standard configuration opens a browser connecting to playground at localhost with port 8080. If you've running playground in a separate virtual environment like e.g. in a docker container, it may be necessary to start the browser manually, determine the VM-/Containers-IP and fill in the address manually in the URL field.

\todo{screenshots of playground}

\textbf{Run Transactions with composer-cli}

Init network transaction:

\begin{lstlisting}[language=bash]
	composer transaction submit -c admin@sardex-open-network -d  '{ "$class": "net.sardex.interlace.InitBlockchain" }'
\end{lstlisting}

The InitBlockchain transaction is setting up some basic accounts as well as demo members to continue with simple transactions right away.

Submit a credit transfer from account a1 to a2 with amount of 800 SRD:

\begin{lstlisting}[language=bash]
	composer transaction submit -c admin@sardex-open-network -d  '{ "$class": "net.sardex.interlace.CreditTransfer", "amount": 800, "senderAccount": "resource:net.sardex.interlace.CCAccount#a1", "recipientAccount": "resource:net.sardex.interlace.CCAccount#a2" }'
\end{lstlisting}

Submit a debit transfer from account a1 to a2 with amount of 200 SRD:

\begin{lstlisting}[language=bash]
	composer transaction submit -c admin@sardex-open-network -d  '{ "$class": "net.sardex.interlace.DebitTransfer", "amount": 200, "senderAccount": "resource:net.sardex.interlace.CCAccount#a1", "recipientAccount": "resource:net.sardex.interlace.CCAccount#a2" }'
\end{lstlisting}

A successful debit transfer creates a PendingTransfer entry with status Pending containing an OTP (one time pad). This OTP can be used by the debitor to confirm the transaction. Thus in the next example "995317396" is used to call a transaction DebitTransferAcknowledge to acknowledge the debit transfer:

\begin{lstlisting}[language=bash]
	composer transaction submit -c admin@sardex-open-network -d  '{ "$class": "net.sardex.interlace.DebitTransferAcknowledge", "transfer": "resource:net.sardex.interlace.PendingTransfer#995317396" }'
\end{lstlisting}

\textbf{The web front-end}

The web front-end currently is a simple web site generated by a yeoman generator provided by the composer-community. The web application can be found in the webapp directory.

In order to get the web application to run properly it is necessary to start-up the whole network and start the REST-server as described in the previous steps.

The web app which is based on AngularJS needs various node.js packages downloaded and installed which is achieved by calling

\begin{lstlisting}[language=bash]
	cd webapp
	npm install
\end{lstlisting}

After that a development server can be started by calling

\begin{lstlisting}[language=bash]
	cd webapp
	npm start
\end{lstlisting}

npm will start a web server at port 4200. If you work locally it also tries to open a browser which is showing the web application, otherwise you'd need start a browser manually and enter the URL by yourself. This is the URL where the server can be reached:

\begin{lstlisting}
http://interlace.chain:4200
\end{lstlisting}

The web page is based on AngularJS and communicates over REST with our previously started REST server facilitating asynchronous AJAX-request.

\section{Technical Details}

\section{Identity Management}

\todo{outline:}
\begin{verbatim}

Sec: Architecture:
  - Sardex Network
  - fabric network  
  - cto-model 
    
Sec: Prototype
    - installation
    - usage
    
Sec: Technical Details:
    - chain-code bits
    - queries
    - acl-file
    - deployment
    - web application
    - docker-compose
    - rest server https://hyperledger.github.io/composer/latest/reference/rest-server

Sec: Identity Management Considerations:
  - permissions
  - https://www.codementor.io/gangachris125/passport-jwt-authentication-for-hyperledger-composer-rest-server-jqfgkoljn
  - passport.js (swagger) www.passportjs.org/docs/
  - user profile strategies

\end{verbatim}
